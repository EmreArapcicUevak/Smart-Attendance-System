\documentclass[a4paper, 12pt]{article}

\usepackage[margin = 1in]{geometry} % for spacing around
\usepackage{graphicx} % for including images in your pdfs
\usepackage{xcolor} % for including colors in your pdf
\usepackage{soul} % for text decoration
\usepackage[utf8]{inputenc} % for encoded text
\usepackage[T1]{fontenc}
\usepackage{setspace} % for setting different line spacings between paragrafs.
\usepackage{enumerate} % for letting us get more detailed enumerate lists
\usepackage{multirow} % to let us combine more rows together
\usepackage{colortbl} % for decorating tables
\usepackage{amsmath} % used for representing more complicated math displays
\usepackage{supertabular}
\usepackage{longtable} % both of these packages are used to making really big tables
\usepackage{wrapfig} % allows us to wrap text around figures
\usepackage{fancyhdr} % for making fancy headers
%\usepackage{bibtex} % for making better bibliographies
\usepackage[pdftex]{hyperref} % for letting us make links
\usepackage{lscape} % Allows us to flip from portrait to landspace
\usepackage{tikz} % for high detailed drawing
\usepackage{multicol} % To put things side by side
\usepackage{rotating} % For rotating objects
% \usepackage{draftwatermark} % For adding watermarks
\usepackage{MnSymbol} % for using multiple symbols
\usepackage{mathtools} % Used for more math symbols
\usepackage{xfrac} % For more complciated fractions and to add derivitives
\usepackage{hyperref} % for hyper links
\usepackage{enumitem} % for better enum lists
\usepackage{tcolorbox} % for adding colored text boxes
\usepackage{bm} % Adding bold text to math inputs
\usepackage{pgfplots} % Used for plotting functions
\usepackage{booktabs} % For better table rules
\usepackage{tabularx} % For adjustable-width tables
\usepackage{eso-pic} % For adding background images
\usepackage{transparent} % For transparency
\usepackage{fontawesome5}
\usepackage{svg}

% Setting up the default image path
\graphicspath{{./Images/}}

% Implementing authro details
\title{}
\author{}
\date{}

% Setting up the fancy page style
\fancypagestyle{customStyle}{
	\lhead{} \chead{} \rhead{}
	\lfoot{
    \href{https://github.com/EmreArapcicUevak/Smart-Attendance-System}{\faGithub \ GitHub} \hspace{0.5cm}
  } \cfoot{\thepage} \rfoot{
    \href{https://trello.com/invite/b/67d55d04d8b354319f1b1db9/ATTI66010e9dfa09b0a174c6bbc4d1da915388F90333/smart-attendance-ius}{Trello \includegraphics[height=10pt]{../GlobalImages/trello-logo.pdf}}
  }
	\renewcommand{\headrulewidth}{0pt}
	\renewcommand{\footrulewidth}{1pt}
}
\pagestyle{customStyle}

% Setting up hyperref options
\hypersetup {
	colorlinks = false,
	citecolor = black,
	filecolor = blue,
	linkcolor = blue,
	urlcolor = blue,
	pdftex
}

\tcbset{
	colback=yellow!10, % Background color
	colframe=red!75!black, % Border color
	width=\textwidth, % Full-width box
	boxrule=0.5mm, % Border thickness
	arc=2mm, % Rounded corners
	fonttitle=\bfseries,
	coltitle=black % Title color
}

% Page layout settings
\geometry{top=1in, bottom=1in, left=1in, right=1in}

% Custom commands


\begin{document}
  
  \begin{titlepage}
    \begin{center}
        % University Logo (Optional)
        \includegraphics[width=0.45\textwidth]{../GlobalImages/Logo.png} % Change filename if needed
        \vspace{1.5cm}

        {\Huge \textbf{Smart Attendance System}}\\
        \vspace{0.5cm}
        {\Large \textbf{Project Specifications}}\\
        
        \vspace{2cm}
        
        {\large \textbf{Subject:} Software Engineering}\\
        
        \vspace{2cm}
        
        \begin{minipage}[t]{0.48\textwidth}
          \textbf{Students:}\\
          {\large Emre Arapčić-Uevak}\\
          {\large Vedad Siljić}\\
          {\large Ismail Dedić}\\
          {\large Amer Jusić}\\
          {\large Faris Hasanbasić}\\
        \end{minipage}
        \hfill
        \begin{minipage}[t]{0.48\textwidth}
          \raggedleft
          \textbf{Professor:} \\
          {\large Mirza Selimović}\\
          
          \vspace{0.5cm}

          \textbf{Instructor:} \\
          {\large Adin Jahić}\\
        \end{minipage}


        
        \vfill  % Pushes everything to the top, so date stays at bottom
        
        \textbf{\today}  % Automatically adds the current date
        
    \end{center}
  \end{titlepage}

  \pagebreak
  \tableofcontents
  \pagebreak

  \section{Introduction}
    This section covers the overview and both the Use Case Diagram and EERD to expalin the project

    \subsection{Project Overview}
      The Smart Attendance System is a modern, data-driven solution designed to streamline attendance tracking and class management in educational institutions. The project consists of three main components:
      \begin{enumerate}
        \item A Web Application - Provides access to attendance data, dashboards, and administrative controls.
        \item A Backend Server - Handles data storage, API communication, and system logic.
        \item Native Mobile Applications (iOS \& Android) - Enables attendance tracking and real-time access to schedules.
      \end{enumerate}
    
      \noindent At the core of the system is a centralized database (more about that in Section~\ref{sec: EER}) that maintains records of the organization's structure, including staff, students, class schedules, tutorials, and labs.
      The backend APIs serve as the communication bridge between the database and the web and mobile applications, ensuring secure and efficient data exchange. \\

      \subsubsection{Key Functionalities}
      The system is designed with two main user roles: students and staff, each with distinct access levels and interfaces.
      \begin{itemize}
        \item \textbf{Student Dashboard (Web \& Mobile)} \\
        Students have a calendar-based view of their classes, allowing them to track their schedule and attendance. If they miss too many sessions, they receive alerts and can view their attendance history through a GitHub-style activity tracker that provides a visual representation of their participation.
        \item \textbf{Staff Dashboard (Web \& Mobile)} \\
        Staff members can view and manage the classes they are responsible for. Attendance tracking is done exclusively on mobile devices, where students present their ID cards to be scanned via the professor's (or any other staff member responsible for the class and or session) phone camera. The system reads the barcode on the student’s ID card, ensuring quick and accurate attendance recording.
        \item \textbf{Administrative Controls (Web Only)} \\
        Unlike the mobile applications, the web platform includes an admin panel that allows the organization's management to create and manage user accounts, oversee class structures, and make necessary modifications to the system.
      \end{itemize}

    \newpage
    \subsection{Extended Entity Relation Diagram} \label{sec: EER}
      
\end{document}